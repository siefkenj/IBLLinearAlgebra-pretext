
\begin{exercises}

	\begin{problist}
		\prob For each of the matrices below, find the geometric and algebraic
		multiplicity of each eigenvalue. \label{PROBMOD16-matrices}
		\begin{enumerate}
			\item $A=\mat{2&0\\-2&1}$

			\item $B=\mat{3&0\\0&3}$

			\item $C=\mat{3&0\\3&0}$

			\item $D=\mat{0&3/2&4\\0&1&0\\-1&1&4}$

			\item $E=\mat{2&1/2&0\\0&1&0\\0&1/2&2}$
		\end{enumerate}% Q1 Solutions

		\begin{solution}

			\begin{enumerate}
				% a


				\item
					\begin{enumerate}
						\item $\lambda=1$, Algebraic: 1,
							Geometric: 1

						\item $\lambda=2$, Algebraic: 1,
							Geometric: 1
					\end{enumerate}% b


				\item
					\begin{enumerate}
						\item $\lambda=3$, Algebraic: 2,
							Geometric: 2
					\end{enumerate}% c


				\item
					\begin{enumerate}
						\item $\lambda=0$, Algebraic: 1,
							Geometric: 1

						\item $\lambda=3$, Algebraic: 1,
							Geometric: 1
					\end{enumerate}% d


				\item
					\begin{enumerate}
						\item $\lambda=1$, Algebraic: 1,
							Geometric: 1

						\item $\lambda=2$, Algebraic: 2,
							Geometric: 1
					\end{enumerate}% e


				\item
					\begin{enumerate}
						\item $\lambda=2$, Algebraic: 2,
							Geometric: 2

						\item $\lambda=1$, Algebraic: 1,
							Geometric: 1
					\end{enumerate}
			\end{enumerate}
		\end{solution}

		\prob For each matrix from question \ref{PROBMOD16-matrices},
		diagonalize the matrix if possible. Otherwise explain why the matrix cannot
		be diagonalized. % Q2 Solutions

		\begin{solution}

			\begin{enumerate}
				\item $\mat{2&0\\0&1}$

				\item $\mat{3&0\\0&3}$

				\item $\mat{3&0\\0&0}$

				\item Not diagonalizable.

				\item $\mat{2&0&0\\0&2&0\\0&0&1}$
			\end{enumerate}
		\end{solution}

		\prob Give an example of a $4\times 4$ matrix with $2$ and $7$
		as its only eigenvalues.
		\begin{solution}

				$\mat{2&0&0&0\\0&2&0&0\\0&0&7&0\\0&0&0&7}$
		\end{solution}

		\prob Can the geometric multiplicity of an eigenvalue ever be $0$?
		Explain.


		\begin{solution}
			No. If $\lambda$ is an eigenvalue of a matrix $A$, then $det(A -
			\lambda I)=0$ and therefore $A - \lambda I$ is not invertible. Specifically,
			$\Nullity(A - \lambda I) \geq 1$ and hence there exists at least one
			eigenvector for the eigenvalue $\lambda$. Therefore the geometric
			multiplicity of $\lambda$ is at least one.
		\end{solution}

		\prob
		\begin{enumerate}
			\item Show that if $\vec v_{1}$ and $\vec v_{2}$ are eigenvectors for
				a matrix $M$ corresponding to different eigenvalues, then $\vec
				v_{1}$ and $\vec v_{2}$ are linearly independent.
				\label{MOD16PROBLININD}

			\item If possible, give an example of a non-diagonalizable $3\times
				3$ matrix where $1$ and $-1$ are the only eigenvalues.

			\item If possible, give an example of a non-diagonalizable $2\times
				2$ matrix where $1$ and $-1$ are the only eigenvalues.
		\end{enumerate}

		\begin{solution}
			\begin{enumerate}
				\item Let $\vec v_{1}$ and $\vec v_{2}$ be eigenvectors for a matrix $M$
					corresponding to distinct eigenvalues $\lambda_{1}$ and
					$\lambda_{2}$ respectively. Let $a,b \in \R$ be such that $a
					\vec v_{1}+ b\vec v_{2}= \vec 0$. Multiplying both sides by
					$M - \lambda_{1} I$,
					we get 
					\[
						b(\lambda_{2}- \lambda_{1}) \vec v_{2}= \vec 0.
					\]
					Since
					$\vec v_{2}$ is an eigenvector, it is nonzero. Hence, either $b=0$
					or $\lambda_{2}- \lambda_{1} = 0$. Since $\lambda_1\neq\lambda_2$
					we know $\lambda_{2}- \lambda_{1} \neq 0$ and so $b=0$.

					We have deduced that $a
					\vec v_{1}= \vec 0$. However, since $\vec v_{1}$ is nonzero (because it is an
					eigenvector), we must have that $a = 0$. This means that $\vec v_{1}$
					and $\vec v_{2}$ are linearly independent.

				\item $\mat{1&1&0\\0&1&0\\0&0&-1}$

				\item This is impossible. Suppose that for some matrix $M$, $\vec v_{1}$
					is an eigenvector corresponding to $1$ and $\vec v_{2}$ is an eigenvector
					corresponding to $-1.$ By \ref{MOD16PROBLININD}, $\vec v_{1}$ and $\vec v_{2}$
					are linearly independent and thus form a basis for $\R^{2}$.
					Since, $\R^{2}$ has a basis consisting of eignvectors of $M$,
					we know $M$ is diagonalizable.
			\end{enumerate}
		\end{solution}
	\end{problist}
\end{exercises}

\begin{exercises}
	\begin{problist}
		\prob Find the complete solution to the following systems.
		\begin{enumerate}
			\item $\systeme[xyzw]{4x+6y+3z-10w=6,5x+2y+z-7w=2,-6x+2y+z+4w=2}$
			\item $\systeme[xyzw]{2x+2y+z=-1,y-4z+2w=3,x-y-3z-4w=5}$
			\item $\systeme{x+y-2z=-5,-4x+y+5z=3}$
			\item $\systeme{3x-2y=-4,x+y+3z=3,-4x+y-3z=1}$
			\item $\systeme{x-y+2z=-1,2x+y+4z=1,3x-4y+3z=-2}$
			\item $\systeme{2x+z=8,x+y+z=4,x+3y+2z=4,3x+2y+4z=9}$
		\end{enumerate}
		\begin{solution}
			\begin{enumerate}
				\item 
				Let
				\[
					X=
					\begin{bmatrix}[cccc|c]
						4 & 6 & 3 & -10 & 6\\
						5 & 2 & 1 & -7 & 2\\
						-6 & 2 & 1 & 4 & 2
					\end{bmatrix}
				\]
				be the augmented matrix corresponding to the system.
				
				By row reduction,
				\[
					\Rref(X)=
					\begin{bmatrix}[cccc|c]
						1 & 0 & 0 & -1 & 0\\
						0 & 1 & 1/2 & -1 & 1\\
						0 & 0 & 0 & 0 & 0
					\end{bmatrix}.
				\]
				
				The third and fourth column of $\Rref(X)$ are free variable columns,
				so we introduce the arbitrary equations $z=t$ and $w=s$ and
				solve the following system in terms of $t$ and $s$:
				\[
					\systeme[xyzw]{x-w=0,y+(1/2)z-w=1,z=t,w=s}.
				\]
				
				Written in vector form, the complete solution is
				\[
					\mat{x\\y\\z\\w} = \matc{s\\1-(1/2)t+s\\t\\s}
					=t\mat{0\\-1/2\\1\\0}+s\mat{1\\1\\0\\1}+\mat{0\\1\\0\\0}.
				\]
				\item 
				Let
				\[
					X=
					\begin{bmatrix}[cccc|c]
						2 & 2 & 1 & 0 & -1\\
						0 & 1 & -4 & 2 & 3\\
						1 & -1 & -3 & -4 & 5
					\end{bmatrix}
				\]
				be the augmented matrix corresponding to the system.
				
				By row reduction,
				\[
					\Rref(X)=
					\begin{bmatrix}[cccc|c]
						1 & 0 & 0 & -2 & 1\\
						0 & 1 & 0 & 2 & -1\\
						0 & 0 & 1 & 0 & -1
					\end{bmatrix}.
				\]
				
				The fourth column of $\Rref(X)$ is a free variable column,
				so we introduce the arbitrary equation $w=t$ and solve the following system
				in terms of $t$:
				\[
					\systeme[xyzw]{x-2w=1,y+2w=-1,z=-1,w=t}.
				\]
				
				Written in vector form, the complete solution is
				\[
				\mat{x\\y\\z\\w} = \matc{1+2t\\-1-2t\\-1\\t} = t\mat{2\\-2\\0\\1}+\mat{1\\-1\\-1\\0}.
				\]
				\item 
				Let
				\[
					X=
					\begin{bmatrix}[ccc|c]
						1 & 1 & -2 & -5\\
						-4 & 1 & 5 & 3
					\end{bmatrix}
				\]
				be the augmented matrix corresponding to the system.
				
				By row reduction,
				\[
					\Rref(X)=
					\begin{bmatrix}[ccc|c]
						1 & 0 & -7/5 & -8/5\\
						0 & 1 & -3/5 & -17/5
					\end{bmatrix}.
				\]
				
				The third column of $\Rref(X)$ is a free variable column, so
				we introduce the arbitrary equation $z=t$ and solve the following system
				in terms of $t$:
				\[
					\systeme{x-7/5z=-8/5,y-3/5z=-17/5,z=t}.
				\]
				
				Written in vector form, the complete solution is
				\[
					\mat{x\\y\\z} = \matc{-8/5+7/5t\\-17/5+3/5t\\t} = t\mat{7/5\\3/5\\1}+\mat{-8/5\\-17/5\\0}.
				\]
				\item 
				Let
				\[
					X=
					\begin{bmatrix}[ccc|c]
						3 & -2 & 0 & -4\\
						1 & 1 & 3 & 3\\
						-4 & 1 & -3 & 1
					\end{bmatrix}
				\]
				be the augmented matrix corresponding to the system.
				
				By row reduction,
				\[
					\Rref(X)=
					\begin{bmatrix}[ccc|c]
						1 & 0 & 6/5 & 2/5\\
						0 & 1 & 9/5 & 13/5\\
						0 & 0 & 0 & 0
					\end{bmatrix}.
				\]
				
				The third column of $\Rref(X)$ is a free variable column, so
				we introduce the arbitrary equation $z=t$ and solve the following system
				in terms of $t$:
				\[
					\systeme{x+6/5z=2/5,y+9/5z=13/5,z=t}.
				\]
				
				Written in vector form, the complete solution is
				\[
					\mat{x\\y\\z} = \matc{2/5-6/5t\\13/5-9/5t\\t} = t\mat{-6/5\\-9/5\\1}+\mat{2/5\\13/5\\0}.
				\]
				\item 
				Let
				\[
					X=
					\begin{bmatrix}[ccc|c]
						1 & -1 & 2 & -1\\
						2 & 1 & 4 & 1\\
						3 & -4 & 3 & -2
					\end{bmatrix}
				\]
				be the augmented matrix corresponding to the system.
				
				By row reduction,
				\[
					\Rref(X)=
					\begin{bmatrix}[ccc|c]
						1 & 0 & 0 & 4/3\\
						0 & 1 & 0 & 1\\
						0 & 0 & 1 & -2/3
					\end{bmatrix}.
				\]
				
				Written in vector form, the complete solution is
				\[
					\mat{x\\y\\z} = \mat{4/3\\1\\-2/3}.
				\]
				\item 
				Let
				\[
					X=
					\begin{bmatrix}[ccc|c]
						2 & 0 & 1 & 8\\
						1 & 1 & 1 & 4\\
						1 & 3 & 2 & 4\\
						3 & 2 & 4 & 9
					\end{bmatrix}
				\]
				be the augmented matrix corresponding to the system.
				
				By row reduction,
				\[
					\Rref(X)=
					\begin{bmatrix}[ccc|c]
						1 & 0 & 0 & 5\\
						0 & 1 & 0 & 1\\
						0 & 0 & 1 & -2\\
						0 & 0 & 0 & 0
					\end{bmatrix}.
				\]
				
				Written in vector form, the complete solution is
				\[
					\mat{x\\y\\z} = \mat{5\\1\\-2}.
				\]
			\end{enumerate}
		\end{solution}

		\prob For each system of linear equations given below: (i) write down
		its augmented matrix, (ii) use row reduction algorithm to determine if it
		is consistent or not, (iii) for each consistent system, give the complete
		solution.
		\begin{enumerate}
			\item \systeme{-10x_1-4x_2+4x_3=28, 3x_1+x_2-x_3=-8, x_1+x_2-\frac{1}{2}x_3=-3}

			\item \systeme{3x_1-2x_2+4x_3=54, 5x_1-3x_2+6x_3=88, x_1=-3}

			\item \systeme{x+2y=5}

			\item \systeme{4x=6, 2x=3}

			\item \systeme{x_1+2x_2+4x_3-3x_4=0, 3x_1+5x_2+6x_3-4x_4=1, 4x_1+5x_2-2x_3+3x_4=3}

			\item \systeme{x_1-x_2+5x_3+x_4=1, x_1+x_2-2x_3+3x_4=3, 3x_1-x_2+8x_3+x_4=5, x_1+3x_2-9x_3+7x_4=5}

			\item \systeme{0x+0y+0z=0}
		\end{enumerate}
		\begin{solution}
			\begin{enumerate}
				\item[(a) i.]
				\[
					\begin{bmatrix}[ccc|c]
						-10 & -4 & 4 & 28\\
						3 & 1 & -1 & -8\\
						1 & 1 & -1/2 & -3
					\end{bmatrix}
				\]
				\item[(a) ii.]
				\begin{align*}
					\displaybreak[0]
					&\begin{bmatrix}[ccc|c]
						-10 & -4 & 4 & 28\\
						3 & 1 & -1 & -8\\
						1 & 1 & -1/2 & -3
					\end{bmatrix}\\
					\rightarrow
					&\begin{bmatrix}[ccc|c]
						1 & 1 & -1/2 & -3\\
						3 & 1 & -1 & -8\\
						-10 & -4 & 4 & 28
					\end{bmatrix}\\
					\rightarrow
					&\begin{bmatrix}[ccc|c]
						1 & 1 & -1/2 & -3\\
						0 & -2 & 1/2 & 1\\
						0 & 6 & -1 & -2
					\end{bmatrix}\\
					\rightarrow
					&\begin{bmatrix}[ccc|c]
						1 & 1 & -1/2 & -3\\
						0 & 1 & -1/4 & -1/2\\
						0 & 6 & -1 & -2
					\end{bmatrix}\\
					\rightarrow
					&\begin{bmatrix}[ccc|c]
						1 & 1 & -1/2 & -3\\
						0 & 1 & -1/4 & -1/2\\
						0 & 0 & 1/2 & 1
					\end{bmatrix}\\
					\rightarrow
					&\begin{bmatrix}[ccc|c]
						1 & 1 & -1/2 & -3\\
						0 & 1 & -1/4 & -1/2\\
						0 & 0 & 1 & 2
					\end{bmatrix}\\
					\rightarrow
					&\begin{bmatrix}[ccc|c]
						1 & 0 & 0 & -2\\
						0 & 1 & 0 & 0\\
						0 & 0 & 1 & 2
					\end{bmatrix}
				\end{align*}
				\item[(a) iii.]
				This system of linear equations is consistent. Its complete solution is
				\[
					\mat{x_1\\x_2\\x_3}=\mat{-2\\0\\2}.
				\]
				\item[(b) i.]
				\[
					\begin{bmatrix}[ccc|c]
						3 & -2 & 4 & 54\\
						5 & -3 & 6 & 88\\
						1 & 0 & 0 & -3
					\end{bmatrix}
				\]
				\item[(b) ii.]
				\begin{align*}
					\displaybreak[0]
					&\begin{bmatrix}[ccc|c]
						3 & -2 & 4 & 54\\
						5 & -3 & 6 & 88\\
						1 & 0 & 0 & -3
					\end{bmatrix}\\
					\rightarrow
					&\begin{bmatrix}[ccc|c]
						1 & 0 & 0 & -3\\
						5 & -3 & 6 & 88\\
						3 & -2 & 4 & 54
					\end{bmatrix}\\
					\rightarrow
					&\begin{bmatrix}[ccc|c]
						1 & 0 & 0 & -3\\
						0 & -3 & 6 & 103\\
						0 & -2 & 4 & 63
					\end{bmatrix}\\
					\rightarrow
					&\begin{bmatrix}[ccc|c]
						1 & 0 & 0 & -3\\
						0 & 1 & -2 & -103/3\\
						0 & -2 & 4 & 63
					\end{bmatrix}\\
					\rightarrow
					&\begin{bmatrix}[ccc|c]
						1 & 0 & 0 & -3\\
						0 & 1 & -2 & -103/3\\
						0 & 0 & 0 & -17/3
					\end{bmatrix}
				\end{align*}
				\item[(b) iii.]
				This system of linear equations is inconsistent.
				\item[(c) i.]
				\[
					\begin{bmatrix}[cc|c]
						1 & 2 & 5
					\end{bmatrix}
				\]
				\item[(c) ii.]
				The augmented matrix of this system of linear equations is already in reduced row echelon form.
				\item[(c) iii.]
				This system of linear equations is consistent. Its complete solution is
				\[
					\mat{x\\y}=t\mat{-2\\1}+\mat{5\\0}.
				\]
				\item[(d) i.]
				\[
					\begin{bmatrix}[c|c]
						4 & 6\\
						2 & 3
					\end{bmatrix}
				\]
				\item[(d) ii.]
				\[
					\begin{bmatrix}[c|c]
					4 & 6\\
					2 & 3
					\end{bmatrix}
					\rightarrow
					\begin{bmatrix}[c|c]
					1 & 3/2\\
					2 & 3
					\end{bmatrix}
					\rightarrow
					\begin{bmatrix}[c|c]
					1 & 3/2\\
					0 & 0
					\end{bmatrix}
				\]
				\item[(d) iii.]
				This system of linear equations is consistent. Its complete solution is $x=3/2$.
				\item[(e) i.]
				\[
					\begin{bmatrix}[cccc|c]
						1 & 2 & 4 & -3 & 0\\
						3 & 5 & 6 & -4 & 1\\
						4 & 5 & -2 & 3 & 3
					\end{bmatrix}
				\]
				\item[(e) ii.]
				\begin{align*}
					\displaybreak[0]
					&\begin{bmatrix}[cccc|c]
						1 & 2 & 4 & -3 & 0\\
						3 & 5 & 6 & -4 & 1\\
						4 & 5 & -2 & 3 & 3
					\end{bmatrix}\\
					\rightarrow
					&\begin{bmatrix}[cccc|c]
						1 & 2 & 4 & -3 & 0\\
						0 & -1 & -6 & 5 & 1\\
						0 & -3 & -18 & 15 & 3
					\end{bmatrix}\\
					\rightarrow
					&\begin{bmatrix}[cccc|c]
						1 & 2 & 4 & -3 & 0\\
						0 & 1 & 6 & -5 & -1\\
						0 & -3 & -18 & 15 & 3
					\end{bmatrix}\\
					\rightarrow
					&\begin{bmatrix}[cccc|c]
						1 & 2 & 4 & -3 & 0\\
						0 & 1 & 6 & -5 & -1\\
						0 & 0 & 0 & 0 & 0
					\end{bmatrix}\\
					\rightarrow
					&\begin{bmatrix}[cccc|c]
						1 & 0 & -8 & 7 & 2\\
						0 & 1 & 6 & -5 & -1\\
						0 & 0 & 0 & 0 & 0
					\end{bmatrix}
				\end{align*}
				\item[(e) iii.]
				This system of linear equations is consistent. Its complete solution is
				\[
					\mat{x_1\\x_2\\x_3\\x_4}=t\mat{8\\-6\\1\\0}+s\mat{-7\\5\\0\\1}+\mat{2\\-1\\0\\0}.
				\]
				\item[(f) i.]
				\[
					\begin{bmatrix}[cccc|c]
						1 & -1 & 5 & 1 & 1\\
						1 & 1 & -2 & 3 & 3\\
						3 & -1 & 8 & 1 & 5\\
						1 & 3 & -9 & 7 & 5
					\end{bmatrix}
				\]
				\item[(f) ii.]
				\begin{align*}
					\displaybreak[0]
					&\begin{bmatrix}[cccc|c]
						1 & -1 & 5 & 1 & 1\\
						1 & 1 & -2 & 3 & 3\\
						3 & -1 & 8 & 1 & 5\\
						1 & 3 & -9 & 7 & 5
					\end{bmatrix}\\
					\rightarrow
					&\begin{bmatrix}[cccc|c]
						1 & -1 & 5 & 1 & 1\\
						0 &  & -2 & 3 & 3\\
						3 & -1 & 8 & 1 & 5\\
						1 & 3 & -9 & 7 & 5
					\end{bmatrix}\\
					\rightarrow
					&\begin{bmatrix}[cccc|c]
						1 & -1 & 5 & 1 & 1\\
						0 & 2 & -7 & 2 & 2\\
						0 & 2 & -7 & -2 & 2\\
						0 & 4 & -14 & 6 & 4
					\end{bmatrix}\\
					\rightarrow
					&\begin{bmatrix}[cccc|c]
						1 & -1 & 5 & 1 & 1\\
						0 & 1 & -7/2 & 1 & 1\\
						0 & 2 & -7 & -2 & 2\\
						0 & 4 & -14 & 6 & 4
					\end{bmatrix}\\
					\rightarrow
					&\begin{bmatrix}[cccc|c]
						1 & -1 & 5 & 1 & 1\\
						0 & 1 & -7/2 & 1 & 1\\
						0 & 0 & 0 & -4 & 0\\
						0 & 0 & 0 & 2 & 0
					\end{bmatrix}\\
					\rightarrow
					&\begin{bmatrix}[cccc|c]
						1 & -1 & 5 & 1 & 1\\
						0 & 1 & -7/2 & 1 & 1\\
						0 & 0 & 0 & 1 & 0\\
						0 & 0 & 0 & 2 & 0
					\end{bmatrix}\\
					\rightarrow
					&\begin{bmatrix}[cccc|c]
						1 & -1 & 5 & 1 & 1\\
						0 & 1 & -7/2 & 1 & 1\\
						0 & 0 & 0 & 1 & 0\\
						0 & 0 & 0 & 0 & 0
					\end{bmatrix}\\
					\rightarrow
					&\begin{bmatrix}[cccc|c]
						1 & 0 & 3/2 & 0 & 2\\
						0 & 1 & -7/2 & 0 & 1\\
						0 & 0 & 0 & 1 & 0\\
						0 & 0 & 0 & 0 & 0
					\end{bmatrix}
				\end{align*}
				\item[(f) iii.]
				This system of linear equations is consistent. Its complete solution is
				\[
					\mat{x_1\\x_2\\x_3\\x_4}=t\mat{-3/2\\7/2\\1\\0}+\mat{2\\1\\0\\0}.
				\]
				\item[(g) i.]
				\[
					\begin{bmatrix}[ccc|c]
						0 & 0 & 0 & 0
					\end{bmatrix}
				\]
				\item[(g) ii.]
				The augmented matrix of this system of linear equations is already in reduced row echelon form.
				\item[(g) iii.]
				This system of linear equations is consistent. Its complete solution is
				\[
					\mat{x\\y\\z}=t\mat{1\\0\\0}+s\mat{0\\1\\0}+r\mat{0\\0\\1}.
				\]
			\end{enumerate}
		\end{solution}

		\prob 
		\label{QSETUPEQUATIONS}
		\begin{enumerate}
			\item Let $\vec v_{1}=\mat{1\\1\\-2\\4}$,
			$\vec v_{2}=\mat{1\\4\\0\\2}$ and
			$\vec v_{3}=\mat{-2\\-2\\4\\-8}$.
			
			Set up and solve a system of linear equations whose solution
			will determine if the vectors $\vec v_{1}$, $\vec v_{2}$ and $\vec
			v_{3}$ are linearly independent.
			
			\item Let $\vec v_{1}=\mat{1\\2\\3}$, $\vec v_{2}=\mat{-2\\1\\0}$
			and $\vec v_{3}=\mat{2\\7\\1}$.
			
			Set up and solve a system of linear equations whose solution
			will determine if the vectors $\vec v_{1}$, $\vec v_{2}$ and $\vec
			v_{3}$ span $\R^{3}$.
			\item Let $\ell_1$ and $\ell_2$ be described in vector form by
			\[
				\overbrace{\vec x=t\mat{1\\3}+\mat{1\\1}}^{\displaystyle \ell_1}
				\quad
				\overbrace{\vec x=t\mat{2\\1}+\mat{3\\4}}^{\displaystyle \ell_2}.
			\]
			Set up and solve a system of linear equations whose solution will
			determine if the lines $\ell_{1}$ and $\ell_{2}$ intersect.
			\item Let $\mathcal{P}_{1}$ and $\mathcal{P}_{2}$
			be described in vector form by
			\[
				\begin{aligned}
					&\mathcal{P}_1:\quad \vec x=t\mat{1\\-1\\0}+s\mat{-1\\-1\\2},\\
					&\mathcal{P}_2:\quad \vec x=t\mat{1\\-1\\1}+s\mat{-1\\3\\-2}+\mat{0\\1\\-1}.
				\end{aligned}
			\]
			Set up and solve a system of linear equations whose solution
			will determine if the planes $\mathcal{P}_{1}$ and
			$\mathcal{P}_{2}$ intersect.
		\end{enumerate}
		\begin{solution}
			\begin{enumerate}
				\item The vectors $\vec v_{1}$, $\vec v_{2}$ and $\vec v_{3}$
				are linearly independent if 
				\[
					x\vec v_1+y\vec v_2+z\vec v_3=\vec 0
				\]
				only has the trivial solution.
				
				This vector equation is equivalent to the system of linear equations
				\[
					\systeme{x+y-2z=0,x+4y-2z=0,-2x+4z=0,4x+2y-8z=0}.
				\]
				
				The complete solution to this system is
				\[
					\mat{x\\y\\z}=t\mat{2\\0\\1}.
				\]
				
				In particular, $(x, y, z)=(2, 0, 1)$ is a non-trivial solution to this
				system, so the vectors $\vec v_{1}$, $\vec v_{2}$ and $\vec v_{3}$
				are linearly dependent.
				
				\item By definition, the vectors $\vec v_{1}$, $\vec v_{2}$, and $\vec v_{3}$
				span $\R^3$ if every vector can be written as a linear combination of $\vec v_1$, $\vec v_2$,
				and $\vec v_3$. In other words, the equation
				\[
					x\vec v_1+y\vec v_2+z\vec v_3=\mat{a\\b\\c}
				\]
				is consistent for all choices of $a$, $b$, and $c$.
				This vector equation is equivalent to the system of linear equations
				\[
					\systeme{x-2y+2z=a,2x+y+7z=b,3x+z=c}.
				\]
				Row reducing, we notice that every column is a pivot column
				and so the system is always consistent. Therefore, $\Span\Set{\vec v_1,\vec v_2,\vec v_3}=\R^3$.
				
				
				\item The lines $\ell_{1}$ and $\ell_{2}$ intersect when their
				$x$ and $y$-coordinates. We first set the parameter variable of $\ell_1$ to $t$ and
				the parameter variable of $\ell_2$ to $s$.
				Then, equating the coordinates
				gives the system of linear equations
				\[
					\systeme[ts]{t-2s=2,3t-s=3}.
				\]
				
				The solution to this system is
				\[
					\mat{t\\s}=\mat{4/5\\-3/5}.
				\]
				
					Since $\vec x=\mat{9/5\\17/5}$ when $t=4/5$ (or $s=-3/5$), the intersection
				of $\ell_{1}$ and $\ell_{2}$ is the point $\mat{9/5\\17/5}$.
				
				\item The planes $\mathcal{P}_{1}$ and $\mathcal{P}_{2}$ intersect
				when their coordinates are equal. Relabeling the parameter variables
				for $\mathcal P_2$ as $q$ and $r$ and equating both vector forms, we get
				the following system of linear equations:
				\[
					\systeme[tsqr]{t-s-q+r=0,-t-s+q-3r=1,2s-q+2r=-1}.
				\]
				
				The complete solution to this system is
				\[
					\mat{t\\s\\q\\r}=u\mat{-2\\-1\\0\\1}+\mat{-1/2\\-1/2\\0\\0}.
				\]
				Thus there are an infinite number of points in $\mathcal P_1\cap \mathcal P_2$.
				
				To find these points, we substitute
				$q=0$ and $r=u$ into the vector form of
				$\mathcal{P}_{2}$. This shows us that $\mathcal P_1\cap \mathcal P_2$ can be
				expressed in vector form by
				\[
					\mat{x\\y\\z}=u\mat{-1\\3\\-2}+\mat{0\\1\\-1}.
				\]
			\end{enumerate}
		\end{solution}
		\prob Presented below some students' arguments for question \ref{QSETUPEQUATIONS}.
		Evaluate whether their reasoning is totally correct, mostly correct, or incorrect.
		If their reasoning is not totally correct, point out what mistake(s) they made and
		how they might be fixed.

		\begin{enumerate}
			\item[(a) i.]
			Consider the vector equation
			\[
				x\vec v_1+y\vec v_2+z\vec v_3=\vec 0
			\]
			where $x, y, z\in\R$.
			
			Since $(x, y, z)=(0, 0, 0)$ is a solution to the equation, the equation
			has the trivial solution. Therefore, the
			vectors $\vec v_{1}$, $\vec v_{2}$ and $\vec v_{3}$ are linearly
			independent.

			\item[(a) ii.]
			Consider the vector equation
			\[
				x\vec v_1+y\vec v_2+z\vec v_3=\vec 0
			\]
			where $x, y, z\in\R$.
			
			Notice that $(x, y, z)=(-2, 0, -1)$ is a solution to the equation.
			Since $y=0$ in this solution, it is a trivial solution and therefore
			the vectors $\vec v_{1}$, $\vec v_{2}$ and $\vec v_{3}$ are
			linearly independent.
			
			\item[(c) i.]	The lines $\ell_{1}$ and $\ell_{2}$ intersect when their $x$ and $y$-coordinates
			are equal. Equating $x$ and $y$-coordinates gives
			\[
				\systeme{t+1=2t+3,3t+1=t+4}.
			\]
			
			This system is equivalent to
			\[
				\systeme{t=-2,2t=3}.
			\]
			
			Since this system is inconsistent, the lines $\ell_{1}$ and $\ell_{2}$ do not
			intersect.
			
			\item[(c) ii.] The lines $\ell_{1}$ and $\ell_{2}$ intersect when their $x$ and $y$-coordinates
			are equal. Equating $x$ and $y$-coordinates gives
			\[
				\systeme{t+1=2s+3,3t+1=s+4}.
			\]
			
			This system is equivalent to
			\[
				\systeme[ts]{t-2s=2,3t-s=3},
			\]
			and the solution is
			\[
				\mat{t\\s}=\mat{4/5\\-3/5}.
			\]
			Therefore the lines $\ell_1$ and $\ell_2$ intersect at $\mat{4/5\\3/5}$.

		\item[(d) i.] 	Notice that
			\[
				\vec x=\mat{1/2\\-1/2\\0}=1/2\mat{1\\-1\\0}+0\mat{-1\\-1\\2}
			\]
			and
			\[
				\vec x=\mat{1/2\\-1/2\\0}=0\mat{1\\-1\\1}-1/2\mat{-1\\3\\-2}+\mat{0\\1\\-1}.
			\]
			
			So, $\vec x=\mat{1/2\\-1/2\\0}$ is a point on $\mathcal{P}_{1}$
			and $\mathcal{P}_{2}$. Therefore the planes $\mathcal{P}_{1}$ and
			$\mathcal{P}_{2}$ intersect.
			
		\item[(d) ii.] Notice that $\vec x=\mat{1\\0\\0}$ is a point in $\mathcal{P}_{2}$,
			but this point is not in $\mathcal P_1$. Therefore the planes do not intersect.

		\end{enumerate}
		\begin{solution}
			\begin{enumerate}
				\item The reasoning is incorrect. The solution
				$(x, y, z)=(0, 0, 0)$ is the trivial solution to the vector
				equation, and it is always a solution to the homogeneous equation
				\[
					\alpha_1\vec v_1+\cdots+\alpha_k\vec v_k=\vec 0
				\]
				no matter what $\vec v_{1}, \dots, \vec v_{k}$ are.
				
				To determine if a set of vectors is linearly independent, we
				need to find out whether the trivial solution is the \emph{only}
				solution to the vector equation. That is, there does not exist
				any non-trivial solution to the vector equation.
				
				\item The reasoning is incorrect. A trivial solution is the
				solution where \emph{all} the variables equal zero, so
				the solution $(x, y, z)=(-2, 0, -1)$ is not a trivial
				solution.
				
				\item The reasoning is incorrect. When equating coordinates
				of two different vector forms,
				the parameter variables needs to be set to different
				letters.
				
				A valid system of linear equations is
				\[
					\systeme[ts]{t-2s=2,3t-s=3}.
				\]
				
				Here we have set the parameter $t$ in the vector form of
				$\ell_{2}$ to $s$.
				
				\item The reasoning is incorrect. The solution $\mat{4/5\\-3/5}$
				to the system of linear equations gives the value of $t$ and
				$s$ at the intersection. To find the intersection of $\ell_{1}$
				and $\ell_{2}$, the value $t=4/5$ or $s=-3/5$ needs to be
				plugged into the vector form of $\ell_{1}$ or $\ell_{2}$.
				
				\item The reasoning is correct. Since
				$\vec x=\mat{1/2\\-1/2\\0}$ is a point on both
				$\mathcal{P}_{1}$ and $\mathcal{P}_{2}$, it is in the intersection
				of $\mathcal{P}_{1}$ and $\mathcal{P}_{2}$, so the planes
				$\mathcal{P}_{1}$ and $\mathcal{P}_{2}$ intersect. However,
				we can not determine if the intersection is a line or a
				plane based on only one point, so we need to set up and solve
				an appropriate system of linear equations.
				
				\item The reasoning is incorrect. 
				Finding one point that is in $\mathcal{P}_{2}$ but not in $\mathcal{P}_{1}$ 
				shows that $\mathcal P_1$ does not intersect $\mathcal P_2$ \emph{at that point}, but
				does not rule out the possibility that $\mathcal P_1$ intersects $\mathcal P_2$
				at a \emph{different} point.
			\end{enumerate}
		\end{solution}
	\end{problist}
\end{exercises}

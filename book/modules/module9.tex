Now that we have a handle on the basics of vectors, we can start thinking about \emph{transformations}.
Transformation (or map) is another word for a function, and transformations show up any time you need to describe
vectors changing. For example, the transformation
\[
	S:\R^2\to\R^2\qquad\text{defined by}\qquad \mat{x\\y}\mapsto\mat{2x\\y}
\]
stretches all vectors in the $\xhat$ direction by a factor of $2$.

\begin{center}
	\begin{tikzpicture}
		\begin{axis}[
		    anchor=origin,
		    name=plot1,
		    disabledatascaling,
		    xmin=-2,xmax=2,
		    ymin=-2,ymax=2,
			xtick={-4,...,4},
			ytick={-2,...,4},
			xticklabels={,,},
			yticklabels={,,},
		    x=.7cm,y=.7cm,
		    grid=both,
		    grid style={line width=.1pt, draw=gray!10},
		    %major grid style={line width=.2pt,draw=gray!50},
		    axis lines=middle,
		    minor tick num=0,
		    enlargelimits={abs=0.5},
		    axis line style={latex-latex},
		    ticklabel style={font=\tiny,fill=white},
		    xlabel style={at={(ticklabel* cs:1)},anchor=north west},
		    ylabel style={at={(ticklabel* cs:1)},anchor=south west}
		]

			\fill[black!50!white, opacity=.1] (-2,-2) rectangle (2,2);
			\begin{scope}[cm={1,0,0,1,(0,0)}, myorange, thick, ->]
				\draw (0,0) -- (1,2);
				\draw (0,0) -- (1,1);
				\draw (0,0) -- (1,0);
				\draw (0,0) -- (1,-1);
				\draw (0,0) -- (1,-2);
				\draw (0,0) -- (2,2);
				\draw (0,0) -- (2,1);
				\draw (0,0) -- (2,0);
				\draw (0,0) -- (2,-1);
				\draw (0,0) -- (2,-2);
				\draw (0,0) -- (0,2);
				\draw (0,0) -- (0,1);
				\draw (0,0) -- (0,-1);
				\draw (0,0) -- (0,-2);
				\begin{scope}[densely dotted, BlueGreen]
					\draw (0,0) -- (-1,2);
					\draw (0,0) -- (-1,1);
					\draw (0,0) -- (-1,0);
					\draw (0,0) -- (-1,-1);
					\draw (0,0) -- (-1,-2);
					\draw (0,0) -- (-2,2);
					\draw (0,0) -- (-2,1);
					\draw (0,0) -- (-2,0);
					\draw (0,0) -- (-2,-1);
					\draw (0,0) -- (-2,-2);
				\end{scope}
			\end{scope}
			\coordinate (A) at (1.25,2.5);
		\end{axis}
		\begin{axis}[
			name=plot2,
			at={($(plot1.east) + (1cm, 0)$)},
			anchor=west,
		    disabledatascaling,
		    xmin=-4,xmax=4,
		    ymin=-2,ymax=2,
			xtick={-4,...,4},
			ytick={-2,...,4},
			xticklabels={,,},
			yticklabels={,,},
		    x=.7cm,y=.7cm,
		    grid=both,
		    grid style={line width=.1pt, draw=gray!10},
		    %major grid style={line width=.2pt,draw=gray!50},
		    axis lines=middle,
		    minor tick num=0,
		    enlargelimits={abs=0.5},
		    axis line style={latex-latex},
		    ticklabel style={font=\tiny,fill=white},
		    xlabel style={at={(ticklabel* cs:1)},anchor=north west},
		    ylabel style={at={(ticklabel* cs:1)},anchor=south west}
		]

			\fill[black!50!white, opacity=.1] (-2,-2) rectangle (2,2);
			\begin{scope}[cm={2,0,0,1,(0,0)}, myorange, thick, ->]
				\draw (0,0) -- (1,2);
				\draw (0,0) -- (1,1);
				\draw (0,0) -- (1,0);
				\draw (0,0) -- (1,-1);
				\draw (0,0) -- (1,-2);
				\draw (0,0) -- (2,2);
				\draw (0,0) -- (2,1);
				\draw (0,0) -- (2,0);
				\draw (0,0) -- (2,-1);
				\draw (0,0) -- (2,-2);
				\draw (0,0) -- (0,2);
				\draw (0,0) -- (0,1);
				\draw (0,0) -- (0,-1);
				\draw (0,0) -- (0,-2);
				\begin{scope}[densely dotted, BlueGreen]
					\draw (0,0) -- (-1,2);
					\draw (0,0) -- (-1,1);
					\draw (0,0) -- (-1,0);
					\draw (0,0) -- (-1,-1);
					\draw (0,0) -- (-1,-2);
					\draw (0,0) -- (-2,2);
					\draw (0,0) -- (-2,1);
					\draw (0,0) -- (-2,0);
					\draw (0,0) -- (-2,-1);
					\draw (0,0) -- (-2,-2);
				\end{scope}
			\end{scope}
			\coordinate (B) at (-1.25,2.5);
		\end{axis}

		\draw[thick] (A) edge[bend left, ->] node[midway,above] {$S$} (B) ;
	\end{tikzpicture}
\end{center}

The transformation
\[
	T:\R^2\to\R^2\qquad\text{defined by}\qquad \mat{x\\y}\mapsto\matc{x+3\\y}
\]
translates all vectors $3$ units in the $\xhat$ direction.

\begin{center}
	\begin{tikzpicture}
		\begin{axis}[
		    anchor=origin,
		    name=plot1,
		    disabledatascaling,
		    xmin=-2,xmax=2,
		    ymin=-2,ymax=2,
			xtick={-4,...,4},
			ytick={-2,...,4},
			xticklabels={,,},
			yticklabels={,,},
		    x=.7cm,y=.7cm,
		    grid=both,
		    grid style={line width=.1pt, draw=gray!10},
		    %major grid style={line width=.2pt,draw=gray!50},
		    axis lines=middle,
		    minor tick num=0,
		    enlargelimits={abs=0.5},
		    axis line style={latex-latex},
		    ticklabel style={font=\tiny,fill=white},
		    xlabel style={at={(ticklabel* cs:1)},anchor=north west},
		    ylabel style={at={(ticklabel* cs:1)},anchor=south west}
		]

			\fill[black!50!white, opacity=.1] (-2,-2) rectangle (2,2);
			\begin{scope}[cm={1,0,0,1,(0,0)}, myorange, thick, ->]
				\draw (0,0) -- (1,2);
				\draw (0,0) -- (1,1);
				\draw (0,0) -- (1,0);
				\draw (0,0) -- (1,-1);
				\draw (0,0) -- (1,-2);
				\draw (0,0) -- (2,2);
				\draw (0,0) -- (2,1);
				\draw (0,0) -- (2,0);
				\draw (0,0) -- (2,-1);
				\draw (0,0) -- (2,-2);
				\draw (0,0) -- (0,2);
				\draw (0,0) -- (0,1);
				\draw (0,0) -- (0,-1);
				\draw (0,0) -- (0,-2);
				\begin{scope}[densely dotted, BlueGreen]
					\draw (0,0) -- (-1,2);
					\draw (0,0) -- (-1,1);
					\draw (0,0) -- (-1,0);
					\draw (0,0) -- (-1,-1);
					\draw (0,0) -- (-1,-2);
					\draw (0,0) -- (-2,2);
					\draw (0,0) -- (-2,1);
					\draw (0,0) -- (-2,0);
					\draw (0,0) -- (-2,-1);
					\draw (0,0) -- (-2,-2);
				\end{scope}
			\end{scope}
			\coordinate (A) at (1.25,2.5);
		\end{axis}
		\begin{axis}[
			name=plot2,
			at={($(plot1.east) + (1cm, 0)$)},
			anchor=west,
		    disabledatascaling,
		    xmin=-2,xmax=5,
		    ymin=-2,ymax=2,
			xtick={-4,...,5},
			ytick={-2,...,4},
			xticklabels={,,},
			yticklabels={,,},
		    x=.7cm,y=.7cm,
		    grid=both,
		    grid style={line width=.1pt, draw=gray!10},
		    %major grid style={line width=.2pt,draw=gray!50},
		    axis lines=middle,
		    minor tick num=0,
		    enlargelimits={abs=0.5},
		    axis line style={latex-latex},
		    ticklabel style={font=\tiny,fill=white},
		    xlabel style={at={(ticklabel* cs:1)},anchor=north west},
		    ylabel style={at={(ticklabel* cs:1)},anchor=south west}
		]

			\fill[black!50!white, opacity=.1] (-2,-2) rectangle (2,2);
			\begin{scope}[cm={1,0,0,1,(3,0)}, myorange, thick, ->]
				\draw (-3,0) -- (1,2);
				\draw (-3,0) -- (1,1);
				\draw (-3,0) -- (1,0);
				\draw (-3,0) -- (1,-1);
				\draw (-3,0) -- (1,-2);
				\draw (-3,0) -- (2,2);
				\draw (-3,0) -- (2,1);
				\draw (-3,0) -- (2,0);
				\draw (-3,0) -- (2,-1);
				\draw (-3,0) -- (2,-2);
				\draw (-3,0) -- (0,2);
				\draw (-3,0) -- (0,1);
				\draw (-3,0) -- (0,-1);
				\draw (-3,0) -- (0,-2);
				\begin{scope}[densely dotted, BlueGreen]
					\draw (-3,0) -- (-1,2);
					\draw (-3,0) -- (-1,1);
					\draw (-3,0) -- (-1,0);
					\draw (-3,0) -- (-1,-1);
					\draw (-3,0) -- (-1,-2);
					\draw (-3,0) -- (-2,2);
					\draw (-3,0) -- (-2,1);
					\draw (-3,0) -- (-2,0);
					\draw (-3,0) -- (-2,-1);
					\draw (-3,0) -- (-2,-2);
				\end{scope}
			\end{scope}
			\coordinate (B) at (-1.25,2.5);
		\end{axis}

		\draw[thick] (A) edge[bend left, ->] node[midway,above] {$T$} (B) ;
	\end{tikzpicture}
\end{center}

\Heading{Images of Sets}

Recall the transformation $S:\R^2\to\R^2$ defined by $\mat{x\\y}\mapsto\mat{2x\\y}$.
If we had a bunch of vectors in the plane, applying $S$ would stretch those vectors in
the $\xhat$ direction by a factor of $2$. For example, let $\mathcal C$ be the circle of radius $1$
centered at $\vec 0$. Applying $S$ to all the vectors that make up $\mathcal C$ produces an ellipse.

\begin{center}
	\begin{tikzpicture}
		\begin{axis}[
		    anchor=origin,
		    name=plot1,
		    disabledatascaling,
		    xmin=-2,xmax=2,
		    ymin=-1,ymax=1,
			xtick={-4,...,4},
			ytick={-2,...,4},
		    x=1cm,y=1cm,
		    grid=both,
		    grid style={line width=.1pt, draw=gray!10},
		    %major grid style={line width=.2pt,draw=gray!50},
		    axis lines=middle,
		    minor tick num=0,
		    enlargelimits={abs=0.5},
		    axis line style={latex-latex},
		    ticklabel style={font=\tiny,fill=white},
		    xlabel style={at={(ticklabel* cs:1)},anchor=north west},
		    ylabel style={at={(ticklabel* cs:1)},anchor=south west}
		]

			\begin{scope}[cm={1,0,0,1,(0,0)}, myorange, very thick, ->]
				\draw (0,0) circle[radius=1];
			\end{scope}
			\node at (.45,.35) {$\mathcal C$};
			\coordinate (A) at (1.25,1.5);
		\end{axis}
		\begin{axis}[
			name=plot2,
			at={($(plot1.east) + (1cm, 0)$)},
			anchor=west,
		    disabledatascaling,
		    xmin=-2,xmax=2,
		    ymin=-1,ymax=1,
			xtick={-4,...,4},
			ytick={-2,...,4},
		    x=1cm,y=1cm,
		    grid=both,
		    grid style={line width=.1pt, draw=gray!10},
		    %major grid style={line width=.2pt,draw=gray!50},
		    axis lines=middle,
		    minor tick num=0,
		    enlargelimits={abs=0.5},
		    axis line style={latex-latex},
		    ticklabel style={font=\tiny,fill=white},
		    xlabel style={at={(ticklabel* cs:1)},anchor=north west},
		    ylabel style={at={(ticklabel* cs:1)},anchor=south west}
		]

			\begin{scope}[cm={2,0,0,1,(0,0)}, mypink, very thick, ->]
				\draw (0,0) circle[radius=1];
			\end{scope}
			\node at (.5,.35) {$S(\mathcal C)$};
			\coordinate (B) at (-1.25,1.5);
		\end{axis}

		\draw[thick] (A) edge[bend left, ->] node[midway,above] {$S$} (B) ;
	\end{tikzpicture}
\end{center}

The operation of applying a transformation to a specific set of vectors and seeing what results is
called taking the \emph{image} of a set.

\SavedDefinitionRender{ImageofaSet}

In plain language, the image of a set $X$ under a transformation $L$ is the set of all outputs of
$L$ when the inputs come from $X$.

If you think of sets in $\R^n$ as black-and-white ``pictures'' (a point is black if it's in the set
and white if it's not), then the image of a set under a transformation is the output after applying the
transformation to the ``picture''.

Images allow one to describe complicated geometric figures in terms of an original figure and a transformation.
For example, let $\mathcal R:\R^2\to\R^2$ be rotation counter clockwise by $30^\circ$
and let $X=\Set{x\xhat+y\yhat\given x,y\in[0,1]}$ be the filled-in unit square. Then, $\mathcal R(X)$ is the filled-in
unit square that meets the $x$-axis at an angle of $30^\circ$. Try describing that using set builder notation!

\begin{center}
	\begin{tikzpicture}
		\begin{axis}[
		    anchor=origin,
		    name=plot1,
		    disabledatascaling,
		    xmin=-0,xmax=1,
		    ymin=-0,ymax=1,
			xtick={-4,...,4},
			ytick={-2,...,4},
		    x=1.5cm,y=1.5cm,
		    grid=both,
		    grid style={line width=.1pt, draw=gray!10},
		    %major grid style={line width=.2pt,draw=gray!50},
		    axis lines=middle,
		    minor tick num=0,
		    enlargelimits={abs=0.5},
		    axis line style={latex-latex},
		    ticklabel style={font=\tiny,fill=white},
		    xlabel style={at={(ticklabel* cs:1)},anchor=north west},
		    ylabel style={at={(ticklabel* cs:1)},anchor=south west}
		]

			\begin{scope}[cm={1,0,0,1,(0,0)}, mypink, ->]
				\fill[opacity=.3] (0,0) rectangle (1,1);
				\draw (0,0) rectangle (1,1);
			\end{scope}
			\node at (.5,.5) {$X$};
			\coordinate (A) at (.75,1.25);
		\end{axis}
		\begin{axis}[
			name=plot2,
			at={($(plot1.east) + (1cm, 0)$)},
			anchor=west,
		    disabledatascaling,
		    xmin=-0,xmax=1,
		    ymin=-0,ymax=1,
			xtick={-4,...,4},
			ytick={-2,...,4},
		    x=1.5cm,y=1.5cm,
		    grid=both,
		    grid style={line width=.1pt, draw=gray!10},
		    %major grid style={line width=.2pt,draw=gray!50},
		    axis lines=middle,
		    minor tick num=0,
		    enlargelimits={abs=0.5},
		    axis line style={latex-latex},
		    ticklabel style={font=\tiny,fill=white},
		    xlabel style={at={(ticklabel* cs:1)},anchor=north west},
		    ylabel style={at={(ticklabel* cs:1)},anchor=south west}
		]

			\begin{scope}[rotate=30, BlueGreen,  ->]
				\fill[opacity=.3] (0,0) rectangle (1,1);
				\draw (0,0) rectangle (1,1);
			\end{scope}
			\node[right] at (0,.6) {$\mathcal R(X)$};
			\coordinate (B) at (-.25,1.25);
		\end{axis}

		\draw[thick] (A) edge[bend left, ->] node[midway,above] {$\mathcal R$} (B) ;
	\end{tikzpicture}
\end{center}

\Heading{Linear Transformations}

Linear algebra's main focus is the study of a special category of transformations: the \emph{linear} transformations.
Linear transformations include rotations, dilations (stretches), shears, and more.

\begin{center}
	\begin{tikzpicture}
		\begin{scope}[cm={1,0,0,1,(0,0)}, black]
			\node[transform shape, black] at (.5,.5) {Box};
			\fill[opacity=.3] (0,0) rectangle (1,1) +(-.5,0)node[opacity=1, above] {Untransformed};
			\draw[thick, ->] (0,0) -- (1,0);
			\draw[thick, densely dotted, ->] (0,0) -- (0,1);
		\end{scope}
		\begin{scope}[cm={1,0,-1,1,(3,0)}, mygreen]
			\node[transform shape, black] at (.5,.5) {Box};
			\fill[mygreen, opacity=.3] (0,0) rectangle (1,1)  +(-.5,0) node[opacity=1, above] {Shear};
			\draw[thick, ->] (0,0) -- (1,0);
			\draw[thick, densely dotted, ->] (0,0) -- (0,1);
		\end{scope}
		\begin{scope}[cm={1, 0, 0, 1,(5,.13)}, Blue]
			\draw[very thick, opacity=.5] (0,0) -- (1.15,.77) +(-.5,0)node[opacity=1, above] {Project};
			\draw[thick, ->] (0,0) -- (.69,.46);
			\draw[thick, densely dotted, ->] (0,0) -- (.46, .31);
		\end{scope}
		\begin{scope}[cm={.936, .352, -.352, .936,(7.5,-.1)}, myorange]
			\node[transform shape, black] at (.5,.5) {Box};
			\fill[opacity=.3] (0,0) rectangle (1,1)  +(-.5,0) node[opacity=1, above] {Rotate};
			\draw[thick, ->] (0,0) -- (1,0);
			\draw[thick, densely dotted, ->] (0,0) -- (0,1);
		\end{scope}
		\begin{scope}[cm={3,0,0,1,(9.4,0)}, mypink]
			\node[transform shape, black] at (.5,.5) {Box};
			\fill[opacity=.3] (0,0) rectangle (1,1)  +(-.5,0) node[opacity=1, above] {Stretch};
			\draw[thick, ->] (0,0) -- (1,0);
			\draw[thick, densely dotted, ->] (0,0) -- (0,1);
		\end{scope}
	\end{tikzpicture}
\end{center}

Linear transformations are an important type of transformation because (i) we have a complete theory of linear transformations
(non-linear transformations are notoriously difficult to understand), and (ii) many non-linear transformations can be approximated
by linear ones\footnote{Just like in one-variable calculus where if you zoom into
a function at a point its graph looks like a line, if you zoom into a (non-linear) transformation at a point, it looks like a linear
one.}. All this is to say that linear transformations are worthy of our study.

Without further ado, let's define what it means for a transformation to be linear.

\SavedDefinitionRender{LinearTransformation}

In plain language, the transformation $T$ is linear, or has the property of \emph{linearity}\index{Linearity},
if it distributes over addition and scalar multiplication.
In other words, $T$ \emph{distributes over linear combinations}.

\begin{example}
	Let $S:\R^2\to\R^2$ and $T:\R^2\to\R^2$ be defined by
	\[
		\mat{x\\y}\stackrel{S}{\mapsto}\mat{2x\\y}\qquad\text{and}\qquad
		\mat{x\\y}\stackrel{T}{\mapsto}\matc{x\\y+4}.
	\]
	For each of $S$ and $T$, determine whether the transformation is linear.
	
	Let $\vec u=\mat{u_1\\u_2}$, $\vec v=\mat{v_1\\v_2}$ be vectors, and let $\alpha$ be a scalar.

	We first consider $\mathcal S$.
	We need to verify that $\mathcal S(\vec u+\vec v)=\mathcal S(\vec u)+\mathcal S(\vec v)$ 
	and $\mathcal S(\alpha\vec u)=\alpha\mathcal S(\vec u)$. 
	
	Computing, we see
	\[
	    \mathcal S(\vec u +\vec v)=\mathcal S\left(\mat{u_1+v_1\\u_2+v_2}\right)=\matc{2u_1+2v_1\\u_2+v_2}=\mat{2u_1\\u_2}+\mat{2v_1\\v_2}=\mathcal S(\vec u) + \mathcal S(\vec v)
	\]
	and
	\[
		\mathcal S(\alpha\vec u)=\mat{2\alpha u_1\\a u_2}=\mathcal S\mat{\alpha u_1\\a u_2}=\alpha\mat{2 u_1\\ u_2}=\alpha\mathcal S(\vec u),
	\]
	and so $\mathcal S$ satisfies all the properties of a linear transformation.
	
	\medskip
	Next we consider $\mathcal T$.
	Notice that $\mathcal T(\vec u+\vec v)=\matc{u_1+v_1\\u_2+v_2+4}$ doesn't 
	look like $\mathcal T(\vec u)+\mathcal T(\vec v)=\matc{u_1+v_1\\u_2+v_2+8}$. 
	Therefore, we will guess that $\mathcal T$ is not linear and look for a counter example. 
	
	Using $\xhat=\mat{1\\0}$ and $\yhat=\mat{0\\1}$, we see
	\[
	    \mathcal T(\xhat +\yhat)=\mathcal T\left(\mat{1\\1}\right)=\mat{1\\5} \neq \mat{1\\4}+\mat{0\\5}=\mathcal T(\xhat) + \mathcal T(\yhat).
	\]
	Since at least one required property of a linear transformation is violated, $\mathcal T$ cannot be a linear transformation.
\end{example}

\Heading{Function Notation vs\mbox{.} Linear Transformation Notation}\index{Linear transformation!notation}

Linear transformations are just special types of functions. In calculus, it is traditional to use lower case
letters for a function and parenthesis ``('' and ``)'' around the input to the function.
\[
	\underbrace{f:\R\to\R}_{\text{a function named $f$}}\qquad \underbrace{f(x)}_{\text{$f$ evaluated at $x$}}
\]
For (linear) transformations, it is traditional to use capital letters to describe the function/transformation
and parenthesis around the input are optional.
\[
	\underbrace{T:\R^n\to\R^m}_{\text{a transformation named $T$}}\qquad \underbrace{T(\vec x)}_{\text{$T$ evaluated at $\vec x$}}
	\qquad
	\underbrace{T\vec x}_{\text{also $T$ evaluated at $\vec x$}}
\]
Since sets are also traditionally written using capital letters, sometimes a font variant is used to when writing the transformation
or the set. For example, we might use a regular $X$ to denote a set and a calligraphic $\mathcal T$ to describe a transformation.

\bigskip

Another difference you might not be used to is that, in linear algebra, we make a careful distinction between
a function and its output. Let $f:\R\to\R$ be a function. In calculus, you might consider the phrases ``the function $f$''
and ``the function $f(x)$'' to both make sense. In linear algebra, the first phrase is valid and the second is \emph{not}. By writing
$f(x)$, we are indicating ``the output of the function $f$ when $x$ is input''. So, properly we should say ``the \emph{number} $f(x)$''.

This distinction might seem pedantic now, but by keeping our functions as functions and our numbers/vectors as numbers/vectors,
we can avoid some major confusion in the future.


\Heading{The ``look'' of a Linear Transformation}

Images under linear transformations have a certain look to them. Based just on the word
\emph{linear} you can probably guess which figure below represents the image of a grid under
a linear transformation.

\begin{center}
	\begin{tikzpicture}
		\begin{axis}[
		    anchor=origin,
		    name=plot1,
		    disabledatascaling,
		    xmin=-2,xmax=2,
		    ymin=-2.1,ymax=2.1,
			xtick={-4,...,4},
			ytick={-2,...,4},
		    x=1cm,y=1cm,
		    grid=both,
		    grid style={line width=.1pt, draw=gray!10},
		    %major grid style={line width=.2pt,draw=gray!50},
		    axis lines=middle,
		    minor tick num=0,
		    enlargelimits={abs=0.5},
		    axis line style={latex-latex},
		    ticklabel style={font=\tiny,fill=white},
		    xlabel style={at={(ticklabel* cs:1)},anchor=north west},
		    ylabel style={at={(ticklabel* cs:1)},anchor=south west}
		]

			\begin{scope}[cm={2,.5,-.5,1,(0,0)}, mypink, thick]
				\addplot[mark=none]{0};
				\addplot[mark=none]{-1};
				\addplot[mark=none]{-2};
				\addplot[mark=none]{1};
				\addplot[mark=none]{2};
				\addplot[mark=none]({1},{x});
				\addplot[mark=none]({0},{x});
				\addplot[mark=none]({-1},{x});
				\addplot[mark=none]({-.5},{x});
				\addplot[mark=none]({.5},{x});
			\end{scope}
		\end{axis}
		\begin{axis}[
			name=plot2,
			at={($(plot1.east) + (1cm, 0)$)},
			anchor=west,
		    disabledatascaling,
		    xmin=-2,xmax=2,
		    ymin=-2.1,ymax=2.1,
			xtick={-4,...,4},
			ytick={-2,...,4},
		    x=1cm,y=1cm,
		    grid=both,
		    grid style={line width=.1pt, draw=gray!10},
		    %major grid style={line width=.2pt,draw=gray!50},
		    axis lines=middle,
		    minor tick num=0,
		    enlargelimits={abs=0.5},
		    axis line style={latex-latex},
		    ticklabel style={font=\tiny,fill=white},
		    xlabel style={at={(ticklabel* cs:1)},anchor=north west},
		    ylabel style={at={(ticklabel* cs:1)},anchor=south west}
		]

			\begin{scope}[BlueGreen, thick]
				\addplot[mark=none, smooth, samples=50]({x},{.5*sin(deg(2*x))});
				\addplot[mark=none, smooth, samples=50]({x},{.5*sin(deg(2*x))+1});
				\addplot[mark=none, smooth, samples=50]({x},{.5*sin(deg(2*x))+2});
				\addplot[mark=none, smooth, samples=50]({x},{.5*sin(deg(2*x))-1});
				\addplot[mark=none, smooth, samples=50]({x},{.5*sin(deg(2*x))-2});
				\addplot[mark=none, smooth, samples=50]({.5*cos(deg(2*x))+0},{x});
				\addplot[mark=none, smooth, samples=50]({.5*cos(deg(2*x))+1},{x});
				\addplot[mark=none, smooth, samples=50]({.5*cos(deg(2*x))+2},{x});
				\addplot[mark=none, smooth, samples=50]({.5*cos(deg(2*x))-1},{x});
				\addplot[mark=none, smooth, samples=50]({.5*cos(deg(2*x))-2},{x});
			\end{scope}
		\end{axis}

		\node[yshift=-.5cm] at (plot1.south) {Linear};
		\node[yshift=-.5cm] at (plot2.south) {Non-linear};

	\end{tikzpicture}
\end{center}

Let's prove some basic facts about linear transformations.

\begin{theorem}
	If $T:\R^n\to\R^m$ is a linear transformation, then $T(\vec 0)=\vec 0$\index{Zero vector ($\vec 0$)}.
\end{theorem}
\begin{proof}
	Suppose $T:\R^n\to\R^m$ is a linear transformation and $\vec v\in \R^n$. We know
	that $0\vec v=\vec 0$, so by linearity we have
	\[
		T(\vec 0)=T(0\vec v)=0T(\vec v)=\vec 0.
	\]
\end{proof}

\begin{theorem}
	If $T:\R^n\to\R^m$ is a linear transformation, then $T$ takes lines to lines (or points).
\end{theorem}
\begin{proof}
	Suppose $T:\R^n\to\R^m$ is a linear transformation and let $\ell\subseteq\R^n$ be the line
	given in vector form by $\vec x=t\vec d+\vec p$. We want to prove that $T(\ell)$, the image of
	$\ell$ under the transformation $T$, is a line or a point.

	By definition, every point in $\ell$ takes the form $t\vec d+\vec p$ for some scalar $t$.
	Therefore, every point in $T(\ell)$ takes the form $T(t\vec d+\vec p)$ for some scalar $t$.
	But, $T$ is a linear transformation, so
	\[
		T(t\vec d+\vec p) = tT(\vec d)+T(\vec p).
	\]
	If $T(\vec d)\neq \vec 0$, then $\vec x=tT(\vec d)+T(\vec p)$ describes a line in vector form
	and so $T(\ell)$ is a line.
	If $T(\vec d)=\vec 0$, then $T(\ell) = \Set{t\vec 0+T(\vec p)\given \text{$t$ is a scalar}}=\Set{T(\vec p)}$
	is a point.
\end{proof}

\begin{theorem}
	If $T:\R^n\to\R^m$ is a linear transformation, then $T$ takes parallel lines to parallel lines
	(or points).
\end{theorem}
\begin{proof}
	Suppose $T:\R^n\to\R^m$ is a linear transformation and 
	let $\ell_1$ and $\ell_2$ be parallel lines. Then, we may describe $\ell_1$ in vector form
	as $\vec x=t\vec d+\vec p_1$ and we may describe $\ell_2$ in vector form as $\vec x=t\vec d+\vec p_2$.
	Note that since the lines are parallel, the direction vectors are the same.

	Now, $T(\ell_1)$ can be described in vector form by
	\[
		\vec x=tT(\vec d)+T(\vec p_1)
	\]
	and $T(\ell_2)$ can be described in vector form by
	\[
		\vec x=tT(\vec d)+T(\vec p_2).
	\]
	Written this way and provided $T(\ell_1)$ and
	$T(\ell_2)$ are actually lines, we immediately see that $T(\ell_1)$ and $T(\ell_2)$ have the same direction
	vectors and hence are parallel.

	If $T(\ell_1)$ is instead a point, then we must have $T(\vec d)=\vec 0$, and so $T(\ell_2)$ must also be a point.
\end{proof}

\begin{theorem}
	If $T:\R^n\to\R^m$ is a linear transformation, then $T$ takes subspaces to subspaces.
\end{theorem}
\begin{proof}
	Let $T:\R^n\to\R^m$ be a linear transformation and let $V\subseteq \R^n$ be a subspace. We need to show
	that $T(V)$ satisfies the properties of a subspace.

	Since $V$ is non-empty, we know $T(V)$ is non-empty.

	Let $\vec x,\vec y\in T(V)$. By definition, there are vectors $\vec u,\vec v\in V$ so that
	\[
		\vec x=T(\vec u)\qquad\text{and}\qquad \vec y=T(\vec v).
	\]
	Since $T$ is linear, we know
	\[
		\vec x+\vec y=T(\vec u)+T(\vec v)=T(\vec u+\vec v).
	\]
	Because $V$ is a subspace, we know $\vec u+\vec v\in V$ and so we conclude $\vec x+\vec y=T(\vec u+\vec v)\in T(V)$.

	Similarly, for any scalar $\alpha$ we have
	\[
		\alpha\vec x=\alpha T(\vec u)=T(\alpha\vec u).
	\]
	Since $V$ is a subspace, $\alpha\vec u\in V$ and so $\alpha\vec x=T(\alpha\vec u)\in T(V)$.
\end{proof}



\Heading{Linear Transformations and Proofs}

When proving things in math, you have all of logic at your disposal, and
that freedom can be combined with creativity to show some truly amazing things.
But, for better or for worse, proving whether or not a transformation is linear
usually doesn't require substantial creativity.

Let $T:\R^n\to\R^n$ be defined by $T(\vec v)=2\vec v$. To show that $T$ is linear,
we need to show that for \emph{all} inputs $\vec x$ and $\vec y$ and for \emph{all}
scalars $\alpha$ we have
\[
	T(\vec x+\vec y)=T(\vec x)+T(\vec y)\qquad\text{and}\qquad T(\alpha\vec x)=\alpha T(\vec x).
\]
But, there are an infinite number of choices for $\vec x$, $\vec y$, and $\alpha$. How can we argue about all
of them at once?

Consider the following proof that $T$ is linear.
\begin{proof}
	Let $\vec x,\vec y\in \R^n$ and let $\alpha$ be a scalar. By applying the definition
	of $T$, we see
	\[
		T(\vec x+\vec y)=2(\vec x+\vec y)=2\vec x+2\vec y = T(\vec x)+T(\vec y).
	\]
	Similarly,
	\[
		T(\alpha\vec x) = 2(\alpha\vec x)=\alpha(2\vec x)=\alpha T(\vec x).
	\]
	Since $T$ satisfies the two properties of a linear transformation, $T$ is a linear
	transformation.
\end{proof}

This proof starts out with ``{\color{mypink}let $\vec x,\vec y\in \R^n$ and let $\alpha$ be a scalar}''.
In what follows, the only properties of $\vec x$ and $\vec y$ we use come from the fact that they're
in $\R^n$ (the domain of $T$) and the only fact about $\alpha$ we use is that it's a scalar. Because
of this, $\vec x$, and $\vec y$ are considered \emph{arbitrary} vectors and $\alpha$ is an
\emph{arbitrary} scalar. Put another way, the argument that followed would work for every single pair 
of vectors $\vec x,\vec y\in \R^n$ and for every scalar $\alpha$.
Thus, by fixing arbitrary vectors at the start of our proof, we are (i) able to argue about
all vectors at once while (ii) having named vectors that we can actually use in equations.


\begin{emphbox}[Takeaway]
	Starting a linearity proof\index{Linear transformation!proving linearity} with ``\emph{let $\vec x,\vec y\in \R^n$ and let $\alpha$ be a scalar}''
	allows you to argue about all vectors and scalars simultaneously.
\end{emphbox}


The proof given above is very typical, and almost every proof of the linearity of a function $T:\R^n\to\R^m$
will look something like
\begin{proof}
	Let $\vec x,\vec y\in \R^n$ and let $\alpha$ be a scalar. By applying the definition
	of $T$, we see
	\[
		T(\vec x+\vec y)=\text{\color{mypink}application(s) of the definition} = T(\vec x)+T(\vec y).
	\]
	Similarly,
	\[
		T(\alpha\vec x) = \text{\color{mypink}application(s) of the definition}=\alpha T(\vec x).
	\]
	Since $T$ satisfies the two properties of a linear transformation, $T$ is a linear
	transformation.
\end{proof}

This isn't to say that proving whether or not a transformation is linear is \emph{easy},
but all the cleverness and insight required appears in the ``{\color{mypink}application(s) of the definition}''
parts.

\bigskip
What about showing a transformation is \emph{not} linear? Here we don't need to show something true for all vectors
and all scalars. We only need to show something is false for \emph{one} pair of vectors or \emph{one} pair of a vector
and a scalar.

When proving a transformation is not linear, we can pick one of the properties of linearity (distribution over
vector addition or distribution over scalar multiplication) and a \emph{single example} where that property fails\footnote{
It's often tempting to argue that the properties of linearity fail for all inputs, but this is a dangerous path! For instance,
if $T(\vec 0)=\vec 0$, then $T(\vec a)=T(\vec a+\vec 0)=T(\vec a)+T(\vec 0)=T(\vec a)$ \emph{regardless} of whether
$T$ is linear or not.}.

\begin{example}
	Let $T:\R^n\to\R^n$ be defined by $T(\vec x)=\vec x+\xhat$. Show that
	$T$ is \emph{not} linear.

	\begin{proof}
		We will show that $T$ does not distribute with respect to scalar multiplication.

		Observe that
		\[
			T(2\vec 0)=T(\vec 0) = \vec e_1\neq 2\vec e_1=2T(\vec 0).
		\]
		Therefore, $T$ cannot be a linear transformation.
	\end{proof}
\end{example}


\Heading{Matrix Transformations}

We already know two ways to interpret matrix multiplication---linear combinations of the columns
and dot products with the rows---and we're about to have a third.

Let $M=\mat{1&2\\-1&1}$. For a vector $\vec v\in \R^2$, $M\vec v$ is another vector in $\R^2$. In this way,
we can think of multiplication by $M$ as a transformation on $\R^2$. Define
\[
	T:\R^2\to\R^2\qquad\text{by}\qquad T(\vec x)=M\vec x.
\]
Because $T$ is defined by a matrix, we call $T$ a \emph{matrix transformation}\index{Matrix!matrix transformation}.
It turns out all matrix transformations are linear transformations and most linear transformations
are matrix transformations\footnote{ If you believe in the axiom of choice and you allow infinitely sized matrices,
every linear transformation can be expressed as a matrix transformation.}. 

When it comes to specifying linear transformations, matrices are heroes, providing a compact notation (just like
they did for systems of linear equations). For example, we could say,
``The linear transformation $T:\R^2\to\R^2$ that doubles the $x$-coordinate and triples the $y$-coordinate'', or
we could say, ``The matrix transformation given by $\mat{2&0\\0&3}$''.

\bigskip

When talking about matrices and linear transformations, we must keep in mind that they are not
the same thing. A matrix is a box of numbers and has no meaning until we give it meaning. 
A linear transformation is a function that inputs vectors and outputs vectors. We can \emph{specify} a linear transformation
using a matrix, but a matrix by itself is \emph{not} a linear transformation\footnote{ Consider the function defined
by $f(x)=2x$. You would never say that the function $f$ is $2$!}.

\begin{emphbox}[Takeaway]
	Matrices and linear transformations are closely related, but they aren't the same thing.
\end{emphbox}

So what are some correct ways to specify a linear transformation using a matrix? 
For a matrix $M$, the following are correct.
\begin{itemize}
	\item The transformation $T$ defined by $T(\vec x)=M\vec x$.
	\item The transformation given by multiplication by $M$.
	\item The transformation induced by $M$.
	\item The matrix transformation given by $M$.
	\item The linear transformation whose matrix is $M$.
\end{itemize}

\Heading{Finding a Matrix for a Linear Transformation}

Every linear transformation from $\R^n$ to $\R^m$ has a matrix\index{Linear transformation!representation in a basis},
and we can use basic algebra to find an appropriate matrix.

Let $T:\R^n\to\R^m$ be a linear transformation. Since $T$ inputs vectors with $n$ coordinates and outputs
vectors with $m$ coordinates, we know any matrix for $T$ must be $m\times n$. The process of finding
a matrix for $T$ can now be summarized as follows: (i) create an $m\times n$ matrix of variables, (ii) use
known input-output pairs for $T$ to set up a system of equations involving the unknown variables, (iii) solve
for the variables.

\begin{example}
	Let $\mathcal T:\R^2\to\R^2$ be defined by $\mathcal T\mat{x\\y}=\matc{2x+y\\x}$. Find a matrix, $M$, for $\mathcal T$.

	Because $\mathcal T$ is a transformation for $\R^2\to\R^2$, $M$ will be a $2\times 2$ matrix. Let
	\[
		M=\mat{a&b\\c&d}.
	\]
	We now need to use input-output pairs to ``calibrate'' $M$. We know
	\[
		\mathcal T\mat{1\\1}=\mat{3\\1}\qquad\text{and}\qquad\mathcal T\mat{0\\1}=\mat{1\\0}.
	\]
	Since $M$ is a matrix for $\mathcal T$, we know $\mathcal T\vec x=M\vec x$ for all $\vec x$, and so
	\[
		M\mat{1\\1}=\mat{a&b\\c&d}\mat{1\\1}=\mat{a+b\\c+d}=\mat{3\\1}
	\]
	and
	\[
		M\mat{0\\1}=\mat{a&b\\c&d}\mat{0\\1}=\mat{b\\d}=\mat{1\\0}.
	\]
	This gives us the system of equations
	\[
		\systeme{a+b=3,\phantom{+}c+d=1,b=1,d=0},
	\]
	and solving this system tells us
	\[
		M=\mat{a&b\\c&d} = \mat{2&1\\1&0}.
	\]
\end{example}

	

		\begin{objectives}
	In this tutorial you will be working with inverses and examining how they connect algebra
			and geometry.

	These problems relate to the following course learning objectives:
			\textit{Translate between algebraic and geometric viewpoints to solve problems}, and
	\textit{clearly and correctly express the mathematical ideas of linear algebra to others,
	and understand and apply logical arguments and definitions that have been written by others}.
		\end{objectives}


\subsection*{Problems}

	\begin{enumerate}
	\item Let $f:\R^n\to\R^n$ be a function and let $A$ be a matrix. Write down the definition of (i) what it means
		for $f$ to be \emph{invertible} and (ii) what it means for $A$ to be invertible.
	
	\item Below are several samples from student answers to a MAT223 exam question relating to solving the matrix equation
		$A\vec x=\vec b$. For each sample, decide whether it is totally correct, mostly correct, mostly incorrect, or
		totally incorrect.
		If it is not totally correct, explain what is wrong, and if possible, how to fix it.

		You may assume the student can correctly compute $A^{-1}$.
		\begin{enumerate}
			\item $A\vec x=\vec b$ $\quad\implies\quad$ $\vec x=\vec bA^{-1}$.
			\item $A\vec x=\vec b$ $\quad\implies\quad$ $\vec x=A^{-1}\vec b$.
			\item $A\vec x=\vec b$ $\quad\implies\quad$ $\vec x=\vec b/A$.
			\item $A\vec x=\vec b$ $\quad\implies\quad$ $\vec x=\tfrac{1}{A}\vec b$.
			\item $A\vec x=\vec b$ $\quad\implies\quad$ $A/\vec b=1/\vec x$. $\quad\implies\quad$ $1/(A/\vec b)=\vec x$.
		\end{enumerate}

	\item Let $R:\R^2\to\R^2$ be rotation counter-clockwise by $30^\circ$; let $D:\R^2\to\R^2$ be reflection across the
		line $y=4x$; let $P:\R^2\to\R^2$ be projection onto the line $y=4x$; and let $S:\R^2\to\R^2$ be the transformation
			that doubles the length of every vector.

			For each transformation, (i) decide if it is invertible, and (ii) describe the inverse-transformation in words.
		\item Let $R$, $D$, $P$, and $S$ be defined as before.
			\begin{enumerate}
				\item Determine the rank of $R$, $D$, $P$, and $S$.
				\item Does rank relate to invertibility? Write down an argument to support your conjecture.
				\item Find matrices for $R$, $D$, $P$, and $S$. How does the invertibility of these
					matrices relate to the invertibility of the original transformations?
					Is this relationship affected by which basis you choose to represent the
					transformation in?
			\end{enumerate}
	\end{enumerate}

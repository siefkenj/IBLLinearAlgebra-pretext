
% WARNING!  Do not type any of the following 10 characters except as directed:
%                &   $   #   %   _   {   }   ^   ~   \
%
%%
%%  default option for pdfx.sty  if not specified on the command-line.
\providecommand{\pdfxopt}{a-1b}
%%
%%  Use  {filecontents}  for the  .xmpdata file before input encoding is specified.
%%
%%%%%%%%%%%%%%%%%%%%%%%%%%%%%%%%%%%%%%%%%%%%%%%%%%%
\begin{filecontents*}{\jobname.xmpdata}
	% a macro definition, used below
	\pdfxEnableCommands{% simple macro definitions can be provided everything expands to characters
	 \def\RossPete{Ross \& Pete}
	 }
	\Title{Linear Algebra (\jobname)}%  *not* set by LaTeX's  \title
	\Author{Jason Siefken\sep et al.}% *not* set by LaTeX's \author
	\Subject{Linear Algebra textbook/workbook}
	\Keywords{linear algebra\sep vectors\sep mathematics\sep textbook}
	\Org{University of Toronto}
	\CreatorTool{LaTeX + pdfx.sty with options \pdfxopt}
	\Copyright{Jason Siefken}
	\WebStatement{https://github.com/siefkenj/IBLLinearAlgebra/}% should be URL to copyright statement on the web
	\CoverDisplayDate{2019}
	\CoverDate{2019-08-014}%  must be in format  YYYY-MM-DD  or  YYYY-MM
	\Doi{0.0.0.0}%
	%
	% setting the color profile, these reproduce the defaults; use your own, if required
	%
	% RGB is used with PDF/A (4 parameters):
	\setRGBcolorprofile{sRGB_IEC61966-2-1_black_scaled.icc}{sRGB_IEC61966-2-1_black_scaled}{sRGB IEC61966 v2.1 with black scaling}{http://www.color.org}
	%
	%  For Adobe Color Profiles, set the directory for your system
	%
	%  e.g.  on Mac OS X
	%  What is it under Windows ?
	%
	\gdef\ColorProfileDir{./common/}
	% 
	%  For available profiles, see file  AdobeColorProfiles.tex
	%  For PDF/X-4p or PDF/X-5pg   see file  AdobeExternalProfiles.tex
	%
	%  Now you can use the macros defined in those files:
	 \FOGRAXXXIX
	%
	% or CMYK is used with  PDF/X (4 parameters)
	% \setCMYKcolorprofile{\ColorProfileDir coated_FOGRA39L_argl.icc}{Coated FOGRA39}{FOGRA39 (ISO Coated v2 300\%\space (ECI))}{http://www.color.org}
\end{filecontents*}

\documentclass{workbook}


% pdfx will set color profile etc. information appropriately, so the pdf renders
% consistently across devices. But, it doesn't work with the xelatex-based tectonics
\usepackage{ifxetex}
	\usepackage[utf8]{inputenc}
\ifxetex
\else
	\usepackage[a-3u]{pdfx}
\fi

%%%
% import all needed packages and macros
%%%
\usepackage[yyyymmdd]{datetime}
\input{common/preamble.tex}

% in non-xelatex engines, hyperref is loaded by `pdfx`. If `pdfx` is not loaded, load it here.
\ifxetex
	\usepackage{hyperref}
\else
\fi

%%%
% Set up the footers to have the correct copyright notices
%%%

\fancypagestyle{siefken}{%
	\rfoot{\footnotesize\it \copyright\,Jason Siefken, 2015--2019 \ \makebox(30,5){\includegraphics[height=1.2em]{by-sa.pdf}}}
	\lfoot{}
	\renewcommand{\headrulewidth}{0pt}
}
\fancypagestyle{iola}{%
	\rfoot{\footnotesize\it \copyright\,IOLA Team \url{iola.math.vt.edu} \ \makebox(30,5){\includegraphics[height=2.2em]{images/iolalogo.png}}}
	\lfoot{}
	\renewcommand{\headrulewidth}{0pt}
}

\DeclareDocumentEnvironment{iola}{o}{%
	\newpage
	\pagestyle{iola}
}{%
	\newpage
}


%%
% Allow hiding of environments
%%
\usepackage{environ}% http://ctan.org/pkg/environ
\makeatletter
\newcommand{\voidenvironment}[1]{%
  \expandafter\providecommand\csname env@#1@save@env\endcsname{}%
  \expandafter\providecommand\csname env@#1@process\endcsname{}%
  \@ifundefined{#1}{}{\RenewEnviron{#1}{}}%
}
\makeatother
% allow pagebreaks that only display in `standard` mode
\newcommand{\displayonlynewpage}{\begin{displayonly}\newpage\end{displayonly}}
% allow pagebreaks that only display in `book` mode
\newcommand{\bookonlynewpage}{\begin{bookonly}\newpage\end{bookonly}}




\setbookoptions{
	twosided = false,
	inline solutions = false,
}


\NewColoredEnvironment{
	name = lesson,
	display name = Lesson,
	banner color = Plum,
	title color = Plum,
	banner on left = true,
	open right = false,
}
\NewColoredEnvironment{
	name = module,
	display name = Module,
	banner color = LimeGreen,
	title color = LimeGreen!70!Green!80!black,
	definition color = Cerulean,
	theorem color = myorange,
}
\NewColoredEnvironment{
	name = appendix,
	display name = Appendix,
	banner color = Green,
}
\NewColoredEnvironment{
	name = tutorial,
	display name = Tutorial,
	banner color = Red,
}






\begin{document}
%%
%% Import definitions from definition.tex; all definitions can be restated multiple times
%%

\input{common/definitions.tex}

%%
%% End Definitions
%%

%\begin{module}
%	\Title{Systems of Linear Equations}
%
%	\input{../modules/appendix1.tex}
%	\input{../modules/appendix1-exercises.tex}
%\end{module}
%
%\begin{module}
%	\Title{Systems of Linear Equations II}
%	
%	\input{../modules/appendix2.tex}
%	\input{../modules/appendix2-exercises.tex}
%\end{module}

%\begin{module}
%	\Title{Computing $2\times 2$ and $3\times 3$ Determinants}
%
%
%	\input{../modules/module14.tex}
%%	\input{../modules/appendix1-exercises.tex}
%
%\end{module}

\begin{module}
	\Title{Matrix Multiplication}


%	\input{../modules/appendix2.tex}
	\input{../modules/module16.tex}

\end{module}

\begin{module}
	\Title{Computing $2\times 2$ and $3\times 3$ Determinants}


%	\input{../modules/appendix3.tex}
	\input{../modules/module16-exercises.tex}

\end{module}

\begin{module}
	\color{Blue}
	\footnotesize
	\PrintExerciseSolutions
\end{module}


%\pagestyle{empty}
%
%	\begin{tutorial}
%		\begin{objectives}
%			In this tutorial you will work on rephrasing problems
%			with mathematical language---this is an essential skill if you ever
%			plan on applying mathematical techniques to the world!
%
%			These problems relate to the following course learning
%			objective: \textit{work independently to understand concepts and procedures that have not
%			been previously explained to you}.
%		\end{objectives}
%
%	%	\bigskip 
%
%
%		\subsection*{Problems}
%
%
%		\begin{enumerate}
%			\item Use vectors, sets, set operations, and set-builder notation
%				to describe the following as subsets of $\R^2$.
%				\begin{enumerate}
%					\item The $x$-axis.
%					\item The corners of a square $S$, which is centered at the origin and whose
%						sides have length 3 and are aligned with the axes.
%					\item The diagonal of $S$ (from before) starting from the lower-left to the upper-right.
%					\item Both diagonals of $S$.
%					\item The line segment from $(2,3)$ to $(4,1)$, including the endpoints.
%					\item The line segment from $(2,3)$ to $(4,1)$, not including the endpoints.
%				\end{enumerate}
%
%			\item Let's make a smiley face!\footnote{ This question is not a joke, and a version of
%				it may show up on your midterm.}
%				\begin{enumerate}
%					\item Describe the lower half of a circle of radius 1 centered at the origin.
%						Call this set $M$ (for mouth!).
%					\item Pick a location for the eyes and describe them as small, filled in
%						circles. Call the left eye $L$ and the right eye $R$.
%					\item Describe the whole face using $M$, $L$, and $R$. Call the face $F$.
%					\item When we draw a set, we usual draw black
%						for points in the set and leave points not in the set white. 
%						Let's draw a reverse-face. Come up with a set $F_R$
%						for a face where the ``skin'' of the face is included in $F_R$, but
%						the eyes and mouth are not.
%				\end{enumerate}
%			\item \emph{Interpolation} is the process of filling in points that might not exist already.
%				It's commonly used when zooming-in or rotating a picture on your computer.  The picture $P$ consists
%				of four colored pixels, \emph{\color{red}red} at $(0,0)$, \emph{\color{green!70!black}green} at $(1,0)$,
%				and \emph{\color{blue}blue} at $(2,0)$ and $(3,0)$. To your brain, what is important about this picture
%				is the \emph{relative spacing} between the colors, not their absolute positions. We will interpolate the color
%				positions for a transformed $P$.
%				\begin{enumerate}
%					\item Give the coordinates of each color if $P$ were translated, going from $(1,4)$ to $(4,4)$.
%					\item Give the coordinates of each color if $P$ were twice as big, going from $(0,0)$ to $(6,0)$.
%					\item Give the coordinates of each color if $P$ were rotated and zoomed, going from $(-1,-1)$ to $(7,0)$.
%				\end{enumerate}
%
%		\end{enumerate}
%
%
%%	\begin{solutions}
%%		\begin{enumerate}
%%			\item \begin{enumerate}
%%					\item $\left\{\mat{x\\y}\in\R^2:y=0\right\}$.
%%					\item $\left\{\mat{-3/2\\3/2},\mat{3/2\\3/2},\mat{3/2\\-3/2},\mat{-3/2\\-3/2}
%%							\right\}$.
%%					\item $\left\{\mat{t\\t}:t\in[-3/2,3/2]\right\}$.
%%					\item $\left\{\mat{t\\t}:t\in[-3/2,3/2]\right\}\cup \left\{\mat{t\\-t}:t\in[-3/2,3/2]\right\}$.
%%					\item $\left\{\vec v: \vec v=\alpha \mat{2\\3}+(1-\alpha)\mat{4\\1}\text{ for some }\alpha\in[0,1]\right\}$.
%%					\item $\left\{\vec v: \vec v=\alpha \mat{2\\3}+(1-\alpha)\mat{4\\1}\text{ for some }\alpha\in(0,1)\right\}$.
%%			\end{enumerate}
%%			\item \begin{enumerate}
%%					\item $M=
%%						\left\{\mat{x\\y}: x^2+y^2 = 1\text{ and }y\leq 0\right\}$.
%%					\item Define $\left\|\mat{x\\y}\right\|=\sqrt{x^2+y^2}$ to
%%						be the length of a vector in $\mathbb R^2$.
%%						Let $\vec l=\mat{-1/2\\1}$ and $\vec r=\mat{1/2\\1}$. Then \[L=\{\vec v:\|\vec v-\vec l\|\leq 1/4\}
%%						\qquad\text{and}\qquad R=\{\vec v:\|\vec v-\vec r\|\leq 1/4\}.\]
%%					\item $F=M\cup L\cup R$.
%%					\item $F_R=\{\vec v:\|\vec v\| \leq 3/2\text{ and }\vec v\notin F\}$.
%%			\end{enumerate}
%%			\item \begin{enumerate}
%%				\item red at $(1,4)$, green at $(2,4)$, and blue at $(3,4)$ and $(4,4)$.
%%				\item red at $(0,0)$, green at $(2,0)$, and blue at $(4,0)$ and $(6,0)$.
%%				\item red at $(-1,-1)$, green at $(5/3,-2/3)$, and blue at $(13/3,-1/3)$ and $(7,0)$.
%%			\end{enumerate}
%%		\end{enumerate}
%%	
%%	\end{solutions}
%%
%%	\begin{instructions}
%%
%%		\subsection*{Learning Objectives} Students need to be able to\ldots
%%		\begin{itemize}
%%			\item Turn geometric descriptions and pictures into equations/formulas/sets
%%				suitable for manipulation with mathematics.
%%
%%			\item Be comfortable enough with set notation and operations to combine 
%%				the operations in new ways.
%%		\end{itemize}
%%
%%
%%		\subsection*{Context} Students in class have gone over sets, set operations,
%%		vectors, linear combinations, vector form of lines and planes, and have just started
%%		span. Some sections may also have covered \emph{restricted} linear combinations, for example
%%		convex combinations. Sections have \emph{not} covered norm notation (i.e., $\|\vec x\|$) or
%%		lengths of vectors in general. However, they all know the Pythagorean theorem from high school.
%%
%%
%%		\subsection*{What to Do} This is the first tutorial of the term, and
%%		it is your chance to win the students over! This is a groupwork tutorial,
%%		but students may not be used to working in groups. 
%%
%%		\begin{itemize}
%%			\item Arranged for group work. Reorganize the desks and chairs
%%				(if possible) to facilitate groups of 3 or 4. Ask
%%				students to form groups of 3 or 4 with other students
%%				nearby. Don't allow larger groups.
%%
%%			\item Begin the tutorial by introducing yourself (your name,
%%				your program of study, and your year). You might
%%				also want to give them some more personal information,
%%				such as where you are from or when you first started liking math.
%%
%%			\item Introduce the structure and purpose of tutorials: students
%%				will be working to (1) better understand concepts
%%				from lecture, (2) practice tackling concepts that
%%				have not been explained in lecture, and (3) effectively
%%				communicate. They can expect to spend most of the
%%				tutorial working in small groups.
%%
%%			\item Emphasize the importance of working with others when
%%				learning mathematics---they should be working with
%%				others in this tutorial \emph{and} outside of
%%				class.
%%		\end{itemize}
%%
%%		This introduction should take no more than 5 minutes.
%%
%%		Next, introduce the learning objectives for the day's tutorial. Explain
%%		that the goal of this tutorial. Their worksheet has the ``formal'' objectives
%%		stated and these instructions have the ``hidden'' objectives. Feel free
%%		to share with them the hidden objectives as well.
%%
%%		Ask the students to pair up and
%%		start working on the problem list. Circulate around the room during
%%		this time and ask groups what they're thinking. They will be tempted
%%		to move quickly through the list without thoroughly checking their
%%		new answers---encourage them to think deeply.
%%
%%		Problem 1 is a straightforward question, but students will struggle starting
%%		with part (d) and especially with (e) and (f). They may have forgotten about unions! Ask
%%		them to review the set operations that they know and be creative. When most people are on
%%		parts (e) and (f), go over parts (a)--(d). Then, let them continue working through number 2.
%%		If most of the class gets stuck at any point, draw the class's attention to the front
%%		of the room and work on the difficult part together.
%%
%%		There are too many problems to finish in 50 minutes and \emph{you should not be going
%%		over the solution to every problem}. Solutions will be posted for the students. The goal
%%		of tutorial is for students to spend time \emph{doing} mathematics with an expert around
%%		to help them if they get stuck. Don't feel any time pressure, even if you only get through 1.5
%%		questions, that's okay!
%%
%%		During the last 6 minutes of class, pick one problem (perhaps a few parts of one problem)
%%		that most groups have at least started, and do this problem as a wrapup. Seeing an expert do the
%%		problem is the student's reward for working so hard.
%%
%%		Notes:
%%		\begin{itemize}
%%			\item Students won't have a good conceptualization of convex combinations which make 1(f) and 3(c).
%%				These problems can also be done by describing a line in vector form (i.e., $\vec x=t\vec d+\vec p$)
%%				and restricting the scalar to get points on the line segment.
%%			\item For 1(e), some students might write $\{\vec x:\vec x\text{ is a convex linear combination
%%				of }\vec p\text{ and }\vec q\}$. Other students might think that this description
%%				is ``mathy'' enough. This description is mathy enough, but we can also expand it
%%				by inserting the definition of \emph{convex linear combination} into the set.
%%			\item Problem 2 is more open-ended than they're used to. Some will get excited about this, and others will
%%				be turned off because it's not a ``plug and chug'' question. Emphasize to them that
%%				they will be hired for their creativity and problem-solving, and not their ability to
%%				answer precisely laid out questions, and this is what we're practicing!
%%			\item Students will be confused what part 2(d) means. If so, take some time at the front of the room 
%%				to make a chalk drawing of a face and a reverse face.
%%				Remember, if you're writing on a chalkboard, black and white are already reversed!
%%		\end{itemize}
%%	\end{instructions}
%
%	\end{tutorial}
%
%\begin{module}
%	\Title{Example Module}
%
%
%
%	\begin{theorem}[Volume Theorem I]
%		For a square matrix $M$, $\det(M)$ is the oriented volume of the parallelepiped
%		($n$-dimensional parallelogram) given by the column vectors of $M$.
%	\end{theorem}
%	\begin{theorem}[Volume Theorem II]
%		For a square matrix $M$, $\det(M)$ is the oriented volume of the parallelepiped
%		($n$-dimensional parallelogram) given by the row vectors of $M$.
%	\end{theorem}
%
%	Module \ref{themod}
%
%	You can use this template to
%	\begin{itemize}
%		\item Draft a module.
%		\item Test-drive some figures.
%		\item Play with colors!
%	\end{itemize}
%
%	You can write definitions:
%	\begin{definition}[Turtle Dove]
%		A \emph{bird} that is also a \emph{turtle}.
%	\end{definition}
%	or reference definitions from {\tt definitions.tex}.
%	\SavedDefinitionRender{SetAddition}
%	
%	\input{modules/module1-exercises.tex}
%
%	\PrintExerciseSolutions
%\end{module}
%\begin{module}
%	\Title{Example Module}
%	\label{themod}
%
%	You can use this template to
%	\begin{itemize}
%		\item Draft a module.
%		\item Test-drive some figures.
%		\item Play with colors!
%	\end{itemize}
%
%	You can write definitions:
%	\begin{definition}[Turtle Dove]
%		A \emph{bird} that is also a \emph{turtle}.
%	\end{definition}
%	or reference definitions from {\tt definitions.tex}.
%	\SavedDefinitionRender{SetAddition}
%	
%	\input{modules/module1-exercises.tex}
%
%	\PrintExerciseSolutions
%\end{module}
%
%\begin{lesson}
%	\Title{Linear Combinations}
%
%	\Heading{Textbook}
%	Module 1
%
%	\Heading{Objectives}
%	\begin{itemize}
%		\item Internalize vectors as geometric objects representing displacements.
%
%		\item Use column vector notation to write vectors.
%
%		\item Relate points and vectors and be able to interpret a point as
%			a vector and a vector as a point.
%
%		\item Solve simple equations involving vectors.
%	\end{itemize}
%
%	\Heading{Motivation} Students have differing levels of experience with vectors.
%	We want to establish a common notation for vectors and use vector notation
%	along with algebra to solve simple questions. E.g., ``How can I get to location
%	$A$ given that I can only walk parallel to the lines $y=4x$ and $y=-x$?''
%
%
%	\begin{annotation}
%		\begin{notes}
%			\begin{itemize}
%			\item
%			We will use the language \emph{component of $\vec v$ in
%			the direction $\vec u$} in the future and it will be a \emph{vector}.
%			For this reason, try to refer to the entries of a column
%			vector as \emph{coordinates} or \emph{entries} instead of components.
%
%			\item
%			Though we will almost exclusively use
%			column vector notation in this course, students should be able to parse
%			questions phrased in terms of row vectors.
%			\end{itemize}
%		\end{notes}
%	\end{annotation}We will use column vector notation and the idea of equating
%	coordinates in order to solve problems.
%
%
%
%\end{lesson}
%
%\begin{iola}
%\section*{The Magic Carpet Ride}
%\addcontentsline{toc}{subsection}{The Magic Carpet Ride}
%
%\question
%\begin{annotation}
%	\begin{goals}
%		\Goal{Hands-on experience with vectors as displacements.}
%		\begin{itemize}
%			\item Internalize vectors as geometric objects representing
%				displacements.
%
%			\item Use column vector notation to write vectors.
%
%			\item Use pre-existing knowledge of algebra to answer vector
%				questions.
%		\end{itemize}
%	\end{goals}
%	\begin{notes}
%
%		\begin{itemize}
%			\item There are many ways to solve this problem.
%				Some students
%				might start with equations. After they use their
%				equations to solve the problem, make them draw a picture
%				and come up with a graphical solution.
%
%			\item When the students start coming up with vector equations,
%				give them the vocabulary of \emph{linear
%				combinations}
%				and \emph{column vector notation}.
%		\end{itemize}
%	\end{notes}
%\end{annotation}
%You are a young adventurer. Having spent most of your time in the mythical city of Oronto,
%	you decide to leave home for the first time. Your parents
%want to help you on your journey, so just before your departure, they give you two
%gifts. Specifically, they give you two forms of transportation: a hover board and
%a magic carpet. Your parents inform you that both the hover board and the magic carpet
%have restrictions in how they operate:
%
%\begin{minipage}{\textwidth}
%	\vspace{.5cm}
%	\begin{wrapfigure}{l}{1in}
%	\vspace{-.8cm}
%	\includegraphics[width=1in]{images/HoverBoard-small.png}
%	\end{wrapfigure}
%
%	We denote the restriction on the hover board's movement by the vector
%	$\mat{3 \\1}$. By this we mean that if
%	the hover board traveled ``forward'' for one hour, it would move along a
%	``diagonal'' path that would result in a displacement of 3 km East and
%	1 km North of its starting location.
%\end{minipage}
%
%\begin{minipage}{\textwidth}
%	\vspace{.5cm}
%	\begin{wrapfigure}{l}{1in}
%	\vspace{-.8cm}
%	\includegraphics[width=1in]{images/MagicCarpet-small.png}
%	\end{wrapfigure}
%
%	We denote the restriction on the magic carpet's movement by the vector
%	$\mat{1 \\2 }$. By this we mean that if the
%	magic carpet traveled ``forward'' for one hour, it would move along a
%	``diagonal'' path that would result in a displacement of 1 km East and
%	2 km North of its starting location.
%\end{minipage}
%
%\lfoot{\footnotesize Drawings by \url{@DavidsonJohnR} (twitter)}
%
%\vspace{10mm}
%
%% Scenario Section
%\textbf{Scenario One: The Maiden Voyage}
%
%Your Uncle Cramer suggests that your first adventure should be to go visit
%the wise man, Old Man Gauss. Uncle Cramer tells you that Old Man Gauss
%lives in a cabin that is 107 km East and 64 km North of your home.
%
%\vspace{5mm}
%
%\textbf{Task:}
%\par
%Investigate whether or not you can use the hover board and the magic
%carpet to get to Gauss's cabin. If so, how? If it is not possible to
%get to the cabin with these modes of transportation, why is that the case?
%
%%\vspace{5mm}
%% As a group, state and explain your answer(s) on the group whiteboard. Use
%% the vector notation for each mode of transportation as part of your
%% explanation and use a diagram or graphic to help illustrate your
%% point(s).
%\end{iola}
%
%

\end{document}
